\documentclass[conference]{IEEEtran}
\IEEEoverridecommandlockouts
% The preceding line is only needed to identify funding in the first footnote. If that is unneeded, please comment it out.
\usepackage{cite}
\usepackage{amsmath,amssymb,amsfonts}
\usepackage{algorithmic}
\usepackage{graphicx}
\usepackage{textcomp}
\usepackage{xcolor}
\usepackage{amsmath}
\def\BibTeX{{\rm B\kern-.05em{\sc i\kern-.025em b}\kern-.08em
    T\kern-.1667em\lower.7ex\hbox{E}\kern-.125emX}}
\begin{document}

\title{Multiple Generative Adversarial Nets}

\author{\IEEEauthorblockN{1\textsuperscript{st} Kaiqi Wang \ 16337233}
\IEEEauthorblockA{\textit{School of Data and Computer Sci.} \\
Sun Yat-sen University, Guangzhou, 510275, China \\
wangkq3@mail2.sysu.edu.cn}
}

%\author{\IEEEauthorblockN{1\textsuperscript{st} Chongxuan Li}
%\IEEEauthorblockA{\textit{Dept. of Comp. Sci \& Tech.} \\
%Tsinghua University, Beijing, 100084, China \\
%licx14@mails.tsinghua.edu.cn}
%\and
%\IEEEauthorblockN{2\textsuperscript{nd} Kun Xu}
%\IEEEauthorblockA{\textit{TNList Lab} \\
%Tsinghua University, Beijing, 100084, China \\
%xu-k16@mails.tsinghua.edu.cn}
%\and
%\IEEEauthorblockN{3\textsuperscript{rd} Jun Zhu}
%\IEEEauthorblockA{\textit{State Key Lab of Intell. Tech. \& Sys.} \\
%Tsinghua University, Beijing, 100084, China \\
%dcszj@mail.tsinghua.edu.cn}
%\and
%\IEEEauthorblockN{4\textsuperscript{th} Bo Zhang}
%\IEEEauthorblockA{\textit{Center for Bio-Inspired Computing Research} \\
%Tsinghua University, Beijing, 100084, China \\
%dcszb@mail.tsinghua.edu.cn}
%}

\maketitle

\begin{abstract}
Triple Generative Adversarial Nets (TGANs), which is based on Generative Adversarial Nets (GANs), have gained great success in image generation and semi-supervised learning (SSL). By adding a new role -- classifier, TGANs make the generator, the classifier and the discrimator simultaneously achieve the state-of-art results among deep generative models. \color{red} We guess that if we add more roles into GANs, there can be significant improvement on the quality of image generation and the accuracy of image identification. \color{black} To verify this guess, we present multiple generative adversarial net (MGAN), which consists of $n$ players -- a generator, a discriminator, a classifier $C_1$ , a classifier $C_2$ of classifier $C_1$, a classifier $C_3$ of classifier $C_2$, $\cdots$, a classifier $C_{n-2}$ of classifier $C_{n-3}$. The generator and the classifier characterize the conditional distributions between images and labels, and the discriminator solely focuses on identifying fake image-label pairs. Our results on various datasets demostrate that MGANs have higher quality of image generation and higher accuracy of image identification. 
\end{abstract}

\begin{IEEEkeywords}
Generative Adversarial Nets, Artificial Intelligence, Deep Learning
\end{IEEEkeywords}

\section{Introduction}
Recently, significant progress has been made on generating realistic images based on Generative Adversarial Nets (GANs). GAN is formulated as a two-player game, where the generator $G$ takes a random noise $z$ as input and produces sample $G(z)$ in the data space while the discriminator $D$ identifies whether a certain sample comes from the true data distribution $p(x)$ or the generator. Both $G$ and $D$ are parameterized as deep neural networks and the training procedure is to solve a minimax problem:

$$
\begin{aligned}
\min_G\max_DU(D, G) = & E_{x \sim p(x)}[\log(D(x))] + \\
& E_{z \sim p_z(z)} [ \log(1 - D(G(z)))] \\
\end{aligned}
$$

We attempt to add more roles to the GANs. We introduce $n - 1$ conditional networks -- $n - 2$ classifiers and a generator to generate pseudo labels given real data and pseudo data given real labels, respectively. To justify the quality of the samples from the conditional networks, we define a single discriminator network which can distinguish whether a data-label pair is from the real labeled dataset or not. The resulting model is called Multi-GAN because it has $n$ roles and $n - 1$ conditional networks between them. 

Overall, our main contribution is getting higher quality of image generation and higher accuracy of image identification. 

%\begin{thebibliography}{00}
%\bibitem{b1} G. Eason, B. Noble, and I. N. Sneddon, ``On certain integrals of Lipschitz-Hankel type involving products of Bessel functions,'' Phil. Trans. Roy. Soc. London, vol. A247, pp. 529--551, April 1955.
%\bibitem{b2} J. Clerk Maxwell, A Treatise on Electricity and Magnetism, 3rd ed., vol. 2. Oxford: Clarendon, 1892, pp.68--73.
%\bibitem{b3} I. S. Jacobs and C. P. Bean, ``Fine particles, thin films and exchange anisotropy,'' in Magnetism, vol. III, G. T. Rado and H. Suhl, Eds. New York: Academic, 1963, pp. 271--350.
%\bibitem{b4} K. Elissa, ``Title of paper if known,'' unpublished.
%\bibitem{b5} R. Nicole, ``Title of paper with only first word capitalized,'' J. Name Stand. Abbrev., in press.
%\bibitem{b6} Y. Yorozu, M. Hirano, K. Oka, and Y. Tagawa, ``Electron spectroscopy studies on magneto-optical media and plastic substrate interface,'' IEEE Transl. J. Magn. Japan, vol. 2, pp. 740--741, August 1987 [Digests 9th Annual Conf. Magnetics Japan, p. 301, 1982].
%\bibitem{b7} M. Young, The Technical Writer's Handbook. Mill Valley, CA: University Science, 1989.
%\end{thebibliography}

\end{document}
