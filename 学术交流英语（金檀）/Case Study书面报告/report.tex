% XeLaTeX

\documentclass{article}
\usepackage{ctex}
\usepackage{xypic}
\usepackage{amsfonts,amssymb}
\usepackage{multirow}
\usepackage{geometry}
\usepackage{graphicx}
\usepackage{listings}
\usepackage{lipsum}
\usepackage{courier}
\usepackage{fancyvrb}
\usepackage{etoolbox}


\linespread{1.2}
\geometry{left=3cm,right=2.5cm,top=2.5cm,bottom=2.5cm}

\makeatletter
\patchcmd{\FV@SetupFont}
  {\FV@BaseLineStretch}
  {\fontencoding{T1}\FV@BaseLineStretch}
  {}{}
\makeatother

\lstset{basicstyle=\small\fontencoding{T1}\ttfamily,breaklines=true}
\lstset{numbers=left,frame=shadowbox,tabsize=4}
%\lstset{extendedchars=false}
\begin{document}

\title{Case study: Gender stereotypes about intellectual ability emerge early and influence children's interests}
\author {Kaiqi Wang}
\date{Dec 28, 2017}
\maketitle



\section{Introduction}

Nowadays, people always associate high-level intellectual ability with men more than women. Therefore, most of physicists and philosophers are men. I am interested about the association between gender stereotypes about intellectual ability and children's interests. I would like to learn more about when and why do children's gender stereotypes about intellectual ability emerge. This research has told me the answer. 

In this research, researchers selected 400 children as a sample, which are mostly come from middle class, majority-white U.S.. Researchers took 4 studies to prove that gender stereotypes about intellectual abilityy emerge early and influence children's interests. 

\paragraph{Study 1} In study 1, they assessed 5-to-7-year-old children's endorssement of the "brilliance = males" stereotype and concluded that girls aged 6 and 7 were significantly less likely than boys to associate brilliance with their own gender.

\paragraph{Study 2} In study 2, they tested whether the drop in girl's evaluation of their gender's intellectual abilities is associated with differences between younger and older girls in their perceptions of their school achievement. They found that their was no significant correlation between girl's perceptions of school achievement and their perceptions of brilliance. 

\paragraph{Study 3} In study 3, they investigated whether children's gendered beliefs about brilliance shape their interests and concluded that young children's emergingnotions about who is likely to be brilliant are one of the factors that guide their decisions about which activities to pursue. 

\paragraph{Study 4} In study 4, they predicted that 5-year-old boys' and girls' interest in novel games would not differ because their ideas about brilliance are not yet differentiated. 

The results suggest a conclusion: Many children assimilate the idea that brilliance is a male quality at a young age. This stereotype begins to shape children's interests as soon as it is acquired and is likely to narrow the range of careers they will one day contemplate. 

\section{Evaluation}

I approve of the conclusion of this passage. When I was in kindergarden, I was not aware of the difference between boys' and girls' intellectual ability. We just played together and talked about what game shall we play next and how to play with this toy, etc. When I grew up and entered primary school, I found that most of boys were interested in Mathematics while most of girls were interested in Chinese and English. There is something that motivate boys to be interested in Maths, and girls to be interested in Chinese and English. In my perspective, Chinese and English made me upset because they require me to recide vocabularies, grammar and passages while Maths made me glad because I could solve the problem immediately and gain superior contentment. I asked some girls for their opinions and they said that they were upset because they couldn't solve these Maths problems. It is actually the difference between boys' and girls' intellectual ability that leads to the gender stereotypes, and these stereotypes affect children's interests. 

In this study, I am not able to find out how to calculate own-gender brilliance score, own-gender niceness score and interest score. We cannot validate that whether the scores are objective to their perspective of own-gender brilliance, own-gender niceness and their interests in smart/try-hard game. We are also not able to re-do this experiment to verify that whether the data is correct or not. 

\section{My Perspective}

Although gender stereotypes about intellectual ability emerge early and is rooted in people's heart, we can defeat it by improving ourselves and keeping confident because outstanding, professional ablity and confidence can make people change their conventional gender stereotypes. On the other hand, we don't need to think about others' impression on us, just follow our hearts and do our best! 

For society, keeping everyone equal is appreciated. We are expected to attribute one's brilliance to his efforts and willpower, instead of gender. 

Women are under-represented at the very top of the business world and only made up 29 per cent of board members in these companies in June 2015. The government should improve the proportion of women on the boards of listed companies. 

In 2014, women’s incomes were more than 13 per cent lower than men’s incomes, if all women and men had worked full-time. One can conclude that women’s work is valued less than men’s work. Sectors dominated by women often have lower income levels. So it is important to eliminate unjustified gender pay differentials. Reintroducing the requirement of annual pay surveys is an urgently needed action. 


\end{document}
















