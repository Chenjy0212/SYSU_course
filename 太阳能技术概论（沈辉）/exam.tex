% XeLaTeX

\documentclass{article}
\usepackage{ctex}
\usepackage{xypic}
\usepackage{amsfonts,amssymb}
\usepackage{multirow}
\usepackage{geometry}
\usepackage{graphicx}
\usepackage{listings}
\usepackage{lipsum}
\usepackage{courier}
\usepackage{fancyvrb}
\usepackage{etoolbox}


\linespread{1.2}
\geometry{left=3cm,right=2.5cm,top=2.5cm,bottom=2.5cm}

\makeatletter
\patchcmd{\FV@SetupFont}
  {\FV@BaseLineStretch}
  {\fontencoding{T1}\FV@BaseLineStretch}
  {}{}
\makeatother

\lstset{basicstyle=\small\fontencoding{T1}\ttfamily,breaklines=true}
\lstset{numbers=left,frame=shadowbox,tabsize=4}
%\lstset{extendedchars=false}
\begin{document}

\title{太阳能技术概论 \ 考试提纲}
\author {Africamonkey}
\maketitle

\section{太阳能概述}

\subsection{什么是广义的太阳能}

广义的太阳能包括太阳能、风能、水能、生物质能,也包括煤炭、石油、天然气等化石能源。狭义的太阳能是指太阳辐射能。

\subsection{什么是太阳能}

太阳辐射能,简称太阳能。太阳主要是由氢气与少量的氦气等组成,太阳内部不断地进行着核聚变反应,温度极高,内部温度可达到 $1 \times 10^7 K$ 以上,表面也达到 $5800 K$ 。气态的太阳表层可以视为吸收系数为 $A=1$ 的黑体。

\subsection{太阳能的主要特点}

可再生、洁净、分布广泛、永不枯竭,总量大。但能量密度低、间隙性、受到气候、季节变化等因素影响,不稳定。

\subsection{简单说明太阳与地球的关系}

太阳半径为 $6.96 \times 10^8$ m, 而地球半径为 $6.38 \times 10^6$ m, 太阳-地球平均距离为 $1.496 \times 10^{11}$ m。在太阳与地球的中心连线上,地球表面某点至太阳的张角仅为 32 分,因此可近似地将太阳光投射到地球上的光线视为平行光束。

地球表面的平均温度基本保持不变, 这源于在太阳表层与地球轨道之间的距离上,太阳作为黑体辐射源与地球保持温度平衡。

\subsection{太阳常数}

太阳常数是 \emph{地球大气层外垂直面上太阳辐射在单位面积的总量} 。太阳常数(大气层平均值)为 $1367 (\pm 7) W / m^2$ 。由于大气层对太阳辐射吸收、散射等造成衰减,地面接收到的太阳辐射强度一般在 $1000 W / m^2$ 以下。

\subsection{我国的太阳能资源分布情况}

最好的地区是西藏、青海、新疆、甘肃、内蒙、云南等。

东北地区、华北、山东、海南等地区属于第二类地区

华中、华南属于中等地区

贵州地区与四川成都地区的阳光资源较差。

\subsection{太阳能利用考虑的两个条件}

1. 太阳能资源丰富

2. 空气洁净度好

\subsection{大气质量}

大气质量(Air Mass)定义为 $AM = \frac{1}{sin( \theta )}$ ,其中 $\theta$ 表示太阳高度角。

AM0 是大气层外的太阳能辐射分布,是太空太阳电池的光谱参照。

AM1 是太阳高度角为 90 度的太阳辐射分布。

AM1.5 是太阳高度角为 41.8 度的太阳辐射分布。

\subsection{太阳辐射光谱分布情况}

太阳辐射能量密度是电磁波频率的函数,在地面上的太阳辐射或太阳光的波长范围主要在紫外与近红外之间,一般为 250 - 2500 nm 。

紫外线 \ 小于 380 nm ,占 9\%

可见光 380 - 780 nm ,占 45\%

红外线 \ 大于 780 nm , 占46\%

\subsection{太阳能转换的基本形式有哪些?主要应用在哪些方面?}

基本原理主要有光热转换,光电转换及光化学能转换。

光热技术主要依据\emph{光热转换原理},应用方面主要有热水器、干燥除湿、光热发电等。

光伏技术主要依据\emph{光电转换原理},主要利用半导体材料发展的光伏发电技术。

植物的光合作用主要是依据\emph{光化学能转换原理},光合作用是地球上最广泛地太阳能利用。

\subsection{什么是光伏效应}

光伏效应:光线照在半导体材料上产生光生载流子现象。
光电效应:光束照射在金属表面会使其发射出电子的物理效应。

\section{半导体物理基础}

\subsection{常用高纯硅的硅源料有哪些? 各有哪些特点?}

半导体硅源主要有 $SiCl_4$ 、$SiH_2Cl_2$ 、$SiHCl_3$ 和 $SiH_4$ 四种。

各种硅源优、缺点比较:

$SiCl_4$ 法温度比 $SiHCl_3$ 高,制得 $SiCl_4$ 氯气消耗量大,现少用;

$SiH_4$ 法由于消耗金属镁等还原剂,以及 $SiH_4$ 法本身易燃、易爆等,在一定程度上受到限制。但此法去除硼杂质很有效,无腐蚀性,生产的硅质量高,多用于外延生长;

$SiH_2Cl_2$ 易燃易爆,得到的硅质量高,多用于外延生长;

$SiHCl_3$ 的沸点比 $SiCl_4$ 低,且易于纯化,此法用得多。

\subsection{高纯多晶Si的制备工艺有那些?}

在半导体工业中就采用这些硅源根据不同工艺来制取多晶硅材料:即 $SiCl_4$ 、 $SiH_2Cl_2$  、 $SiHCl_3$ 还原法和 $SiH_4$ 热分解法,反应式如下:

$SiCl_4+ 2H_2 = Si + 4HCl (1100-1200 °C)$

$SiHCl_3 + H_2= Si +3HCl (900-1100 °C)$

$SiH_4 = Si + 2H_2­ (800-1000 °C)$

\subsection{制备多晶硅西门子工艺的技术路线}

冶金硅粉与氯化氢反应生成三氯氢硅(无色液体)

通过蒸馏提纯至 11N 以上

加上氢气通入返原炉加热至 $1100$ 度以上

将硅沉积在炉内的U型籽晶条上

最后形成高纯多晶硅棒

这是半导体器件与晶体硅太阳电池所用的硅片的主要原料

\subsection{什么是半导体?半导体材料的特性?}

自然界的物质按导电性的强弱,可分为导体、半导体和绝缘体三类。

它们的电阻率变化范围为:导体 $< 10^{-4} \Omega \cdot cm$ ;绝缘体 $> 10^8 \Omega \cdot cm$;半导体 $10^{-4}-10^8 \Omega \cdot cm$ 。

半导体具有以下特性:

\subsubsection{掺杂特性}

掺入微量杂质可引起载流子浓度变化,从而明显改变半导体的导电能力。此外,在同一种材料中掺入不同类型的杂质,可得到不同导型的材料(p或n型);

\subsubsection{温度特性}

与金属不同,本征(纯净)半导体具有负的温度系数,即随着温度升高,电阻率下降。但掺杂半导体的温度系数可正可负,要具体分析。

\subsubsection{环境特性}

光照、电场、磁场、压力和环境气氛等也同样可引起半导体导电能力变化。如硫化镉薄膜,其暗电阻为数十兆欧姆,而受光照时的电阻可下降到数十千欧姆(光电导效应)

\subsection{何为p型半导体,n型半导体?}

空穴导电为主的半导体为p型半导体,电子导电为主的半导体为n型半导体。

\subsection{工业上制备硅单晶有哪两种生产工艺?}

一种是直拉法, 直拉法是半导体单晶生长用的最多的一种晶体生长技术。

第二种是区熔法,不用坩埚,采用高频电感加热,通过上下移动加热区,将固体多晶硅棒制成单晶硅,也可以用于提纯晶体硅。

\subsection{简单描述单晶Si的直拉法工艺步骤}

将高纯多晶硅块料加入石英坩埚,并加入掺杂剂,加热至熔化,将籽晶与熔融硅表面接触,慢慢提拉并旋转,通过缩颈、放肩、等径生长,收尾等步骤,即可生成与籽晶相同晶向的单晶硅棒。

    引晶:通过电阻加热,将装在石英坩埚中的多晶硅熔化,并保持略高于硅熔点的温度,将籽晶浸入熔体,然后以一定速度向上提拉籽晶并同时旋转引出晶体;
    
    缩颈:生长一定长度的缩小的细长颈的晶体,以防止籽晶中的位错延伸到晶体中;
    
    放肩:将晶体控制到所需直径;
    
    等径生长:根据熔体和单晶炉情况,控制晶体等径生长到所需长度;
    
    收尾:直径逐渐缩小,离开熔体;
    
    降温:降底温度,取出晶体,待后续加工
    
\subsection{简单描述单晶Si的区熔法工艺步骤}

先制成所要求直径的多晶硅棒,在下端接上一个籽晶,通过高频电感加热,使得与籽晶接触处熔化,然后从下向上提拉,直至多晶硅棒形成如同籽晶一样的晶向的单晶硅棒。

\subsection{直拉单晶Si的和区熔单晶Si的质量区别}

直拉:需要石英坩埚,氧含量高,用于太阳电池,价格低

区熔:不用石英坩埚,氧含量低,用于集成电路、价格高

\subsection{电子级和太阳能级半导体材料的纯度范围}

电子级9N以上,用于半导体器件,太阳能级6-9N,用于太阳电池。

\subsection{晶体硅所用的掺杂元素有哪些?}

p型半导体:常用B, Ga等元素
 
n型半导体:最常用P元素  

\subsection{硅片制造工艺流程}

直拉法制备单晶硅碇,或浇铸法制备多晶硅碇,采用带锯切成硅砖,然后通过多线切割切成硅片。单晶硅片边长多为150 mm,厚度180-200 µm。

多晶硅片边长多为156 mm,厚度180-200 µm。

\subsection{硅片的主要电学技术指标}

单晶硅片的电阻率 1-3 Ωcm,少子寿命10 µs

多晶硅片的电阻率0.5-3 Ωcm,少子寿命1-2 µs

\subsection{p-n结制备工艺有哪些}

扩散、合金化、离子注入、外延等工艺。半导体工业主要采用离子注入,而晶体硅太阳电池主要采用热扩散工艺。

\subsection{导体、绝缘体与半导体的能带}

    导体、绝缘体与半导体的导电能力的差别在于它们的能带结构不同的缘故。导体,如金属的价带与导带之间没有禁带,两者或是重叠,或是价带能级没有被电子填满,而有许多空能级。因此,即使在常温下,靠热激发也有大量的自由电子参与导电。所以,金属的电阻率很低。( $< 10^{-4}$ Ωcm) 
    
    半导体与绝缘体的价带与导带之间都有一个禁带。但是半导体的禁带宽度较这窄 (小于3 eV),随温度升高,价带顶附近的电子容易通过热激发跃迁到导带成为导电电子。其电阻率高于金属,但比绝缘体要小,且随温度升高而减少 ($10^{-4}-10^8$ Ωcm) 
    
    绝缘体的禁带宽度比半导体宽的多(大于3 eV),所以一般情况下,其导带上电子极少,即绝缘体如玻璃、陶瓷、橡胶和塑料等不导电( $> 10^8$ Ωcm)

\subsection{什么是非平衡载流子?}

当处于热平衡的半导体受到外界作用时,将产生比平衡态时多出来的载流子,称为非平衡载流子。光、电等作用也会产生非平衡载流子。

\subsection{什么是载流子的复合与寿命?}

非平衡态载流子从产生到消失(或复合)的平均存在时间定义为非平衡态载流子的平均寿命。寿命是半导体材料的重要参数。一般说寿命往往是指非平衡少数载流子寿命,简称少子寿命

\subsection{什么是载流子的复合中心?}

直接复合: 空穴与电子直接相遇 (砷化镓); 

间接复合,通过复合中心 (硅、锗);复合中心: 一为重金属, 二是晶体缺陷。 

\subsection{p-n结的内建电场是如何形成的?}

一般对于晶体硅太阳电池来说,p-n结是通过热扩散形成的,p型硅片通过采用三氯氧磷进行热扩散,就可在硅片表面形成n型层,即形成p-n结。这样由于存在浓度差别,空穴向p型区、电子向n型区相互扩散,在硅片表面除处形成一个只有施主或只有受主的区域,即空间电荷区,从而形成一个内建电场,这个内建电场阻止空穴与电子的进一步扩散,达到一个动态平衡。

\section{太阳电池原理与工艺}

\subsection{简单说明晶体硅太阳电池的制备工艺过程}

\emph{必考}

硅片清洗、制成绒面结构、扩散制备p-n结、利用PECVD沉积氮化硅减反射薄膜、丝网印刷正负电极、烧结、最后检测分类。

\subsection{简单描述晶体硅太阳电池的工作原理}

\emph{必考}

  在光照下,硅片表面产生电子-空穴对,在内建电场的作用下,空穴向p型区、电子向n型区迁移,从而形成电流。
  
\subsection{太阳电池标准测试条件有哪些?}

\emph{必考}

光强 $1000 W/m^2$ 、大气质量 AM1.5、温度25度

\subsection{硅片表面湿化学处理的目的}

一是清除污染物,二是在硅片表面形成陷光结构。对于单晶硅片通过碱腐蚀形成金字塔结构;对于多晶硅通过酸腐蚀形成凹坑结构。

\subsection{硅片表面金子塔构造的作用}

由于金字塔结构形成,可使得光线两次以上接触硅片,从而增加硅片对光线的吸收效果。

\subsection{[111]和[100]硅片,哪个可通过化学腐蚀法实现表面织构化处理,为何?}

 硅片的(111)晶面为密排面,化学键多;相比之下,(100)晶面化学键少,易于腐蚀。选用晶向为[100]硅片,由于(100)晶面腐蚀速率快,最后留下在硅片表面留下许多(111)组成的所谓金字塔结构,即可达到限制光线反射效果。

\subsection{氮化硅薄膜在多晶硅太阳电池中的作用}

减少光线反射,并通过氢离子与硅的悬挂键结合,对硅片可以起到表面钝化作用。

\subsection{晶体硅太阳电池表面金属化的工艺}

 晶体硅电池的金属化:丝网印刷在背面印上银铝浆电极、烘烤,然后印刷铝背场、烘烤,再印正面银电极,最后烧结。
 
\subsection{太阳光谱的分布范围和晶体硅、非晶硅太阳电池的响应范围?}

太阳光谱一般在300 - 2500 nm范围,晶体硅电池可利用到1100 nm,而非晶硅电池只利用到800 nm。

\subsection{写出太阳电池的填充系数和效率基本公式}

记最大工作点电流为 $I_m$ ,最大工作点电压为 $V_m$ ,短路电流为 $I_{sc}$ ,开路电压为 $V_{oc}$ 。

$FF = \frac{V_mI_m}{V_{oc}I_{sc}}$

$\eta = \frac{I_mV_m}{P_{Licht}} = \frac{FFI_{sc}V_{oc}}{P_{Licht}}$

\subsection{说明选择性发射结太阳电池结构与原理}

 硅片表面实现选择性掺杂:电极下面重掺杂,接受光线处可疑轻掺杂。这样改善短波响应,同时可疑减少串联电阻。
 
\subsection{为何HIT太阳电池可以实现高效率}

非晶硅与单晶硅结合,非晶硅可见光谱范围相应好;单晶硅在近红外光谱响应好,低温工艺,材料损伤小,可实现双面电池。

\subsection{薄膜太阳电池有哪些种类与特点}

薄膜电池有a-Si, CdTe, CIGS等。

直接带隙材料,需要很薄的材料就可吸收光线;衬底材料选择玻璃、金属、塑料等,能够实现大面积生产;容易实现柔性电池;容易实现叠层电池;弱光性能好。

\subsection{非晶硅薄膜与电池的有哪些特点?}

  a-Si禁带宽度为1.7 eV, 通过掺B 或掺P可得到p型a-Si或n型a-Si;

非晶硅掺C, 可得到a-SiC, 禁带宽度 > 2.0 eV(宽带隙),掺 Ge ,可得到 a-SiGe 禁带宽度1.7-1.4 eV (窄带隙);

    在太阳光谱的可见光范围内,非晶硅的吸收系数比晶体硅大将近一个数量级,其本征吸收系数高达$10^5 cm^{-1}$;

    非晶硅太阳电池光谱响应的峰值与太阳光谱的峰值接近;

由于非晶硅材料的本征吸收系数很大, $1 \mu m$ 厚度就能充分吸收太阳光,厚度不足晶体硅的 $\frac{1}{100}$,可节省昂贵的半导体材料。

非晶硅薄膜利用光谱范围局限于可见光,效率较低。

\subsection{什么是S-W效应}

非晶硅及其合金的光暗电导率随光照时间加长而减少, 经200度退火2小时可恢复原状。

\subsection{提高太阳电池效率的原理与方法}

光学、电学两方面考虑:

光学:光谱分割利用;聚光;光子能量上、下转换;全背面电极

电学:叠层电池;异质结电池

\subsection{如何评价太阳电池性能?}


重要参数是能量转换效率,即单位面积上输出电量与输入光能的比值。

提高太阳电池转换效率主要从两个方面进行。

一是光学方面,尽可能提高太阳电池对入射光的吸收,以产生更多的光生载流子;

另外是电学方面,尽量减少光生载流子在电池体内及表面处的复合,同时减少各种上、下电极的电阻损耗,使更多的电能能够输出到外部负载。

\newpage

\section{往年考试试题}

考试方式:闭卷

考试时间: 120 分钟

\subsection{判断题(共 10 题, 30 分)}

1. 光热转换与光电转换的基本原理是一样的。

2. 非晶硅电池的效率比晶体硅电池要高,因为非晶硅电池对太阳光谱的可见光部分有很好的吸收利用。

3. 大气质量 AM0 是指大气层外太阳光谱强度和分布。

4. 像金属一样,本征半导体具有正的温度系数,即随着温度升高,电阻率上升。

5. p型半导体中多数载流子是电子,n型半导体中多数载流子是空穴。

6. 直拉单晶 Si 氧含量高,主要用于太阳电池;区熔单晶 Si 氧含量低,主要用于集成电路。

7. 绝缘体的价带和导带之间没有禁带,或是价带能级没有被电子填满,有许多空能级。

8. p型硅片通过采用三氯氧磷进行热扩散,就可在硅片表面形成n型层,即形成p-n结。

9. 太阳光谱的主要能量分布在 300-2500 nm 范围,晶体硅太阳电池发电可利用的范围为 300-1100 nm 。

10. 晶体硅太阳电池的电极下面重掺杂,形成选择性发射极(SE),可以改善短波响应,同时减少串联电阻。

\subsection{问答题(共 5 题, 30分)}

1. 高纯多晶硅采用西门子工艺,请描述主要工艺过程。

2. 晶体硅太阳电池的主要工艺有哪些?

3. 晶体硅同质结太阳电池p-n结的内建电场是如何形成的?

4. 薄膜太阳电池有哪些种类与特点?

5. 太阳电池国际标准测试(STC)条件有哪些?

\subsection{小论文(共 2 题,选做 1 题,40分)}

1. 如何提高晶体硅太阳电池的转换效率?请谈谈你的观点。

2. 关于光伏发电应用,请谈谈你的想法与建议。

\end{document}































